\documentclass{article}
\usepackage[utf8]{inputenc}
\usepackage[margin=1in,footskip=0.25in]{geometry}
\usepackage[backend=biber, style=authoryear]{biblatex}
\usepackage{hyperref}
\addbibresource{annobibli.bib}

\begin{document}
\title{Masting associated functional traits and seed traits}
\author{Xiaomao Wang}

\maketitle

\section*{Question: What are some functional traits and seed traits are associated with masting? How will that potentially change the reproduction success under the climate change context?}

\section{Traits selection}

There are many hypotheses on mast seeding, including predator satiation, pollination efficiency, environmental prediction, weather cues and resource budget model. We could select traits that help them fit the hypothesis.\\
\textbf{Predator satiation: }\\
	{Size size}\\
	{Seed nutrient content}\\
	{Dispersal potential}\\
	{Seed coat permeability}\\
	{Seasonality of seed release}\\
	{Seed defences}\\
\textbf{Pollinatoin efficiency: }\\
	{Flowering duration}\\
\textbf{Environmental prediction: }\\
	{Seed nutrient content}\\
	{Seed coat thickness}\\
\textbf{Resource matching, resource budget: }\\
	{Longevity}\\
	{Dormancy}\\
	{Leaf phenology}\\
	{leaf longevity}\\
\textbf{weather cues: }\\

\section{Literature Reviews}

\subsection{Overview}

Give Context and brief argument for the need for progress in this research area and what you propose to do, in simple, non-technical terms (for the benefit of non-technical experts). Also briefly emphasize the impact and outcomes from thus work in tangible terms, such as societal benefits
\subsection{Annotated Bibliography}

\begin{itemize}
    \item \fullcite{fernandez2019nutrient}
    \item \fullcite{journe2023evolution}
    \item \fullcite{pearse2020biogeography}
    \item \fullcite{qiu2023masting}
    \item \fullcite{vacchiano2021natural}
\end{itemize}

\printbibliography


\end{document}
