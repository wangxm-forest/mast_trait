% !Rnw encoding = UTF-8
\documentclass{article}
\usepackage{graphicx}
\usepackage{Sweave}
\usepackage{float}
\usepackage[top=1.00in, bottom=1.0in, left=1.1in, right=1.1in]{geometry}

\title{Mast-Trait}
\author{Xiaomao Wang}
\date{\today}

\begin{document}
\Sconcordance{concordance:mastTrait.tex:mastTrait.Rnw:%
1 54 1}


\maketitle


\section*{Hypotheses}
\subsection*{Predator Satiation}

\begin{itemize}
\item Dispersal mode: Animal-dispersed species may be more likely to mast, or
the mechanisms behind masting could differ depending on dispersal mode.
\end{itemize}

\begin{figure}[H]
\centering
\includegraphics[width=\textwidth]{../output/figures/dispersalmode.pdf}
\end{figure}
\newpage
\begin{itemize}
\item \textbf{Seed size:} Among animal-dispersed species, larger-seeded species might be more prone to masting.
\end{itemize}

\begin{figure}[H]
\centering
\includegraphics[width=\textwidth]{../output/figures/seedWeights.pdf}
\includegraphics[width=\textwidth]{../output/figures/seedSize.pdf}
\includegraphics[width=\textwidth]{../output/figures/fruitSize.pdf}
\end{figure}
\newpage
\begin{itemize}
\item \textbf{Seed dormancy:} Species with dormant seeds are expected to be more likely to mast.
\end{itemize}

\begin{figure}[H]
\centering
\includegraphics[width=\textwidth]{../output/figures/seedDormancy.pdf}
\end{figure}

\begin{itemize}
\item \textbf{Nutrient content:} Species with more nutritious seeds may be more likely to mast.
\end{itemize}

\begin{figure}[H]
\centering
\includegraphics[width=\textwidth]{../output/figures/oilContent.pdf}
\end{figure}
\newpage
% ------------------------------------------

\subsection*{Pollination Coupling}

\begin{itemize}
  \item Wind-pollinated species are expected to mast more frequently.
\end{itemize}
\begin{figure}[H]
\centering
\includegraphics[width=\textwidth]{../output/figures/pollination.pdf}
\end{figure}

\begin{itemize}
  \item Monoecious species may be more likely to mast.
\end{itemize}
\begin{figure}[H]
\centering
\includegraphics[width=\textwidth]{../output/figures/reproductiveType.pdf}
\end{figure}
\begin{itemize}
  \item Species with longer flowering periods may be more likely to mast.
\end{itemize}
\begin{figure}[H]
\centering
\includegraphics[width=\textwidth]{../output/figures/flowerDuration.pdf}
\end{figure}

% ------------------------------------------
\newpage
\subsection*{Resource Matching}

\begin{itemize}
  \item Leaf longevity: Species with long-lived leaves are expected to mast more frequently.
\end{itemize}

\begin{figure}[H]
\centering
\includegraphics[width=\textwidth]{../output/figures/leafLongevity.pdf}
\end{figure}

\begin{itemize}
  \item Drought tolerance: May go either way — tolerant species can accumulate resources; intolerant species respond to resource fluctuations.
\end{itemize}

\begin{figure}[H]
\centering
\includegraphics[width=\textwidth]{../output/figures/droughtTolerance.pdf}
\end{figure}
\newpage
% ------------------------------------------
\section*{Data visualization}
I calculated the mean and standard errors for the continuous traits in the raw dataset, with number of the data.
\begin{figure}[H]
\centering
\includegraphics[width=\textwidth]{../output/figures/meanSE.pdf}
\end{figure}


Here's the version of raw data scattering on the plot
\begin{figure}[H]
\centering
\includegraphics[width=\textwidth]{../output/figures/meanSERaw.pdf}
\end{figure}


\newpage
% ------------------------------------------
\section*{Results}
I ran the phyloglm, including the phylogeny variation.Here I reproted the direct model results in a table:
\begin{figure}[H]
\centering
\includegraphics[width=\textwidth]{../output/phyloglmResultsTable.pdf}
\end{figure}

\subsection{Dispersal mode}
\textbf{Conifers}\\
\begin{itemize}
\item Species being abiotic dispersed have an 82\% probability of being strong masting species.
\item Species being biotic dispersed and both have a lower probability, but this difference is not statistically.
\item Overall, the model provides no evidence that dispersal mode affects whether a species is strong masting or not.
\end{itemize}
\textbf{Angiosperms}\\
\begin{itemize}
\item Species being abiotic dispersed have an 32\% probability of being strong masting species.
\item Species being biotic dispersed and both have a higher probability, but this difference is not statistically.
\item Overall, the model provides no evidence that dispersal mode affects whether a species is strong masting or not.
\end{itemize}
\subsection{Pollination mode}
\textbf{Conifers}\\
\begin{itemize}
\item All the conifers in our dataset are wind pollinated, and they are mostly strong masting species.
\end{itemize}
\textbf{Angiosperm}\\
\begin{itemize}
\item Animal pollinated species only have a relatively low probability (20\%) of being strong masting species.
\item Wind pollinated species have asignificantly higher probability (59\%) of being strong masting species.
\item Species pollinated by both animal and wind have an intermediate change (47\%) of being strong masting species, but is not significantly different from animal pollinated ones.
\item Overall, pollination mode is an important predictor of whether species has strong masting pattern, because wind pollination increases the odds of strong masting compared to animal pollination.
\end{itemize}
\subsection{Seed dormancy}
\textbf{Conifers}\\
\begin{itemize}
\item Species without dormant seeds have 80\% probability of being strong masting species.
\item Species with dormant seeds have higher probability (89\%), but this difference is not significant.
\item Overall, the model provides no evidence that seed dormancy affects whether a species is strong masting or not.
\end{itemize}
\textbf{Angiosperm}\\
\begin{itemize}
\item Species without dormant seeds have 39\% probability of being strong masting species.
\item Species with domrmant seeds have a very similar probability (40\%) of being strong masting species, but the difference is not significant.
\item Overall, the model provides no evidence that seed dormancy affects whether a species is strong masting or not.
\end{itemize}
\subsection{Reproductive type}
\textbf{Conifers}\\
\begin{itemize}
\item Dioecious species have a higher probability (70\%) of being strong masting species compared to monoecious species (44\%), but the difference is not significant.
\item The very low alpha (0.007) suggests little phylogenetic signal in this trait-masting relationship.
\item Overall, the model provides no evidence that reproductive type affects whether a species is strong masting or not.
\end{itemize}
\textbf{Angiosperm}\\
\begin{itemize}
\item Dioecious species have 28\% probability of being strong masting species.
\item Monoecious species have higher probability (49\%), but this difference is not significant. Polygamous species have similar probability (26\%), but the difference is not significant.
\item The relatively low alpha (0.11) suggests low to moderate phylogenetic signal in this trait-masting relationship.
\item Overall, the model provides no evidence that reproductive type affects whether a species is strong masting or not.
\end{itemize}
\subsection{Drought tolerance}
\textbf{Conifers}\\
\begin{itemize}
\item High drought tolerance species have a high probability (91\%) of being strong masting species.
\item Moderate and low drought tolerance species have lower or higher probabilities (68\% and 96\% respectively), but differences are not significant.
\item Overall, the model provides no evidence that drought tolerance affects whether a species is strong masting or not.
\end{itemize}
\textbf{Angiosperm}\\
\begin{itemize}
\item High drought tolerance species have a low probability (44\%) of being strong masting species.
\item Moderate and low drought tolerance species both have lower probabilities (40\% and 44\% respectively), but differences are not significant.
\item The relatively low alpha (0.11) suggests low to moderate phylogenetic signal in this trait-masting relationship.
\item Overall, the model provides no evidence that drought tolerance affects whether a species is strong masting or not.
\end{itemize}
\subsection{Seed weight}
\textbf{Conifers}\\
\begin{itemize}
\item Seed weight has no significant effect on strong masting pattern
\item The moderate alpha (0.43) suggests moderate phylogenetic signal in this trait-masting relationship.
\end{itemize}
\textbf{Angiosperm}\\
\begin{itemize}
\item Higher seed weight significantly increases chance of being strong masting species, seed weight is a strong predictor of masting.
\item The relatively low alpha (0.063) suggests low phylogenetic signal in this trait-masting relationship.
\end{itemize}
\subsection{Fruit size}
\textbf{Conifers}\\
\begin{itemize}
\item Fruit size has no significant effect on strong masting pattern.
\item The moderate alpha (0.43) suggests moderate phylogenetic signal in this trait-masting relationship.
\end{itemize}
\textbf{Angiosperm}\\
\begin{itemize}
\item Fruit size has no significant effect on strong masting pattern.
\item The relatively low alpha (0.17) suggests low phylogenetic signal in this trait-masting relationship.
\end{itemize}
\subsection{Seed size}
\textbf{Conifers}\\
\begin{itemize}
\item Seed size has no significant effect on strong masting pattern.
\item The moderate alpha (0.41) suggests moderate phylogenetic signal in this trait-masting relationship.
\end{itemize}
\textbf{Angiosperm}\\
\begin{itemize}
\item Seed size has no significant effect on strong masting pattern.
\item The relatively low alpha (0.08) suggests low phylogenetic signal in this trait-masting relationship.
\end{itemize}
\subsection{Oil content}
\textbf{Conifers}\\
\begin{itemize}
\item Not enough data available for conifers.
\end{itemize}
\textbf{Angiosperm}\\
\begin{itemize}
\item Oil content has no significant effect on strong masting pattern.
\item The relatively low alpha (0.14) suggests low phylogenetic signal in this trait-masting relationship.
\end{itemize}
\subsection{Leaf longevity}
\textbf{Conifers}\\
\begin{itemize}
\item Leaf longevity has no significant effect on strong masting pattern.
\item The moderate alpha (0.42) suggests moderate phylogenetic signal in this trait-masting relationship.
\end{itemize}
\textbf{Angiosperm}\\
\begin{itemize}
\item Leaf longevity has no significant effect on strong masting pattern.
\item The relatively low alpha (0.006) suggests low phylogenetic signal in this trait-masting relationship.
\end{itemize}
\newpage
I also ran the common glm, using conifer and angiosperm as a fixed effect in the model, and I modified the results for better visualization:

For the categorical traits, I calculated the probability (of being a strong masting species) for each level:
\begin{figure}[H]
\centering
\includegraphics[width=\textwidth]{../output/figures/glmCat.pdf}
\end{figure}
\newpage
For the continuous traits, I just presented the effect sizes, with the star indicating a p-value smaller than 0.05:
\begin{figure}[H]
\centering
\includegraphics[width=\textwidth]{../output/figures/glmCon.pdf}
\end{figure}

\end{document}
