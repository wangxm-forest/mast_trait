% !Rnw encoding = UTF-8
\documentclass{article}
\usepackage{Sweave}
\usepackage{float}
\usepackage{graphicx}
\usepackage{natbib}
\usepackage{amsmath}
\usepackage{hyperref}
\usepackage{indentfirst}
\usepackage[top=1.00in, bottom=1.0in, left=1.1in, right=1.1in]{geometry}

\title{Mast-Trait}
\author{Xiaomao Wang}
\date{\today}

\begin{document}
\Sconcordance{concordance:mastTrait.tex:mastTrait.Rnw:%
1 54 1}


\maketitle


\section*{Hypotheses}
\subsection*{Predator Satiation}

\begin{itemize}
\item Dispersal mode: Animal-dispersed species may be more likely to mast, or
the mechanisms behind masting could differ depending on dispersal mode.
\end{itemize}

\begin{figure}[H]
\centering
\includegraphics[width=\textwidth]{../output/figures/dispersalmode.pdf}
\end{figure}
\newpage
\begin{itemize}
\item \textbf{Seed size:} Among animal-dispersed species, larger-seeded species might be more prone to masting.
\end{itemize}

\begin{figure}[H]
\centering
\includegraphics[width=\textwidth]{../output/figures/seedWeights.pdf}
\includegraphics[width=\textwidth]{../output/figures/seedSize.pdf}
\includegraphics[width=\textwidth]{../output/figures/fruitSize.pdf}
\end{figure}
\newpage
\begin{itemize}
\item \textbf{Seed dormancy:} Species with dormant seeds are expected to be more likely to mast.
\end{itemize}

\begin{figure}[H]
\centering
\includegraphics[width=\textwidth]{../output/figures/seedDormancy.pdf}
\end{figure}

\begin{itemize}
\item \textbf{Nutrient content:} Species with more nutritious seeds may be more likely to mast.
\end{itemize}

\begin{figure}[H]
\centering
\includegraphics[width=\textwidth]{../output/figures/oilContent.pdf}
\end{figure}
\newpage
% ------------------------------------------

\subsection*{Pollination Coupling}

\begin{itemize}
  \item Wind-pollinated species are expected to mast more frequently.
\end{itemize}
\begin{figure}[H]
\centering
\includegraphics[width=\textwidth]{../output/figures/pollination.pdf}
\end{figure}

\begin{itemize}
  \item Monoecious species may be more likely to mast.
\end{itemize}
\begin{figure}[H]
\centering
\includegraphics[width=\textwidth]{../output/figures/reproductiveType.pdf}
\end{figure}
\begin{itemize}
  \item Species with longer flowering periods may be more likely to mast.
\end{itemize}
\begin{figure}[H]
\centering
\includegraphics[width=\textwidth]{../output/figures/flowerDuration.pdf}
\end{figure}

% ------------------------------------------
\newpage
\subsection*{Resource Matching}

\begin{itemize}
  \item Leaf longevity: Species with long-lived leaves are expected to mast more frequently.
\end{itemize}

\begin{figure}[H]
\centering
\includegraphics[width=\textwidth]{../output/figures/leafLongevity.pdf}
\end{figure}

\begin{itemize}
  \item Drought tolerance: May go either way — tolerant species can accumulate resources; intolerant species respond to resource fluctuations.
\end{itemize}

\begin{figure}[H]
  \centering
  \includegraphics[width=\textwidth]{../output/figures/droughtTolerance.pdf}
\end{figure}
\newpage
% ------------------------------------------
\section*{Data visualization}

\begin{figure}[H]
  \centering
  \includegraphics[width=\textwidth]{../output/figures/meanSE.pdf}
  \caption{Mean and standard errors for the continuous traits (fruit size, leaf longevity, oil content, seed size, and seed weight) in the original dataset, with number of the data point available.}
\end{figure}


\begin{figure}[H]
  \centering
  \includegraphics[width=\textwidth]{../output/figures/meanSERaw.pdf}
  \caption{Raw data for the continuous traits (fruit size, leaf longevity, oil content, seed size, and seed weight) in the original dataset, with all the data points scattering.}
\end{figure}


\newpage
% ------------------------------------------
\section*{Results for Binary Response Variable}
I ran the phyloglm, including the phylogeny variation.Here I reproted the direct model results in a table:
\begin{figure}[H]
\centering
\includegraphics[width=\textwidth]{../output/phyloglmResultsTable.pdf}
\end{figure}

The probability was calculated using the inverse-logit transformation:
\[
\Pr(Y = 1) = \frac{1}{1 + e^{-(\alpha + \beta x)}}.
\]


\subsection{Dispersal mode}
\textbf{Conifers}\\
\begin{itemize}
\item Species being abiotic dispersed have an 82\% probability of being strong masting species.
\item Species being biotic dispersed and both have a lower probability (67\% and 61\% respectively), but this difference is not statistically.
\item \textbf{The low alpha (0.02) suggests strong phylogenetic signal in this trait-masting relationship.}
\item \textbf{Overall, the model provides no evidence that dispersal mode affects whether a conifer species is strong masting or not.}
\end{itemize}
\textbf{Angiosperms}\\
\begin{itemize}
\item Species being abiotic dispersed have an 32\% probability of being strong masting species.
\item Species being biotic dispersed and both have a higher probability (45\% for both of them), but this difference is not statistically.
\item \textbf{The low alpha (0.06) suggests strong phylogenetic signal in this trait-masting relationship.}
\item \textbf{Overall, the model provides no evidence that dispersal mode affects whether an angiosperm species is strong masting or not.}
\end{itemize}
\subsection{Pollination mode}
\textbf{Conifers}\\
\begin{itemize}
\item All the conifers in our dataset are wind pollinated, and they are mostly strong masting species.
\end{itemize}
\textbf{Angiosperm}\\
\begin{itemize}
\item Animal pollinated species only have a relatively low probability (20\%) of being strong masting species.
\item Wind pollinated species have a significantly higher probability (59\%) of being strong masting species.
\item Species pollinated by both animal and wind have an intermediate change (47\%) of being strong masting species, but is not significantly different from animal pollinated ones.
\item \textbf{The relatively moderate alpha (0.35) suggests moderate phylogenetic signal in this trait-masting relationship.}
\item \textbf{Overall, pollination mode is an important predictor of whether an angiosperm species has strong masting pattern, because wind pollination increases the odds of strong masting compared to animal pollination.}
\end{itemize}
\subsection{Seed dormancy}
\textbf{Conifers}\\
\begin{itemize}
\item Species without dormant seeds have 80\% probability of being strong masting species.
\item Species with dormant seeds have higher probability (89\%), but this difference is not significant.
\item \textbf{The moderate alpha (0.35) suggests moderate phylogenetic signal in this trait-masting relationship.}
\item \textbf{Overall, the model provides no evidence that seed dormancy affects whether a conifer species is strong masting or not.}
\end{itemize}
\textbf{Angiosperm}\\
\begin{itemize}
\item Species without dormant seeds have 39\% probability of being strong masting species.
\item Species with domrmant seeds have a very similar probability (40\%) of being strong masting species, but the difference is not significant.
\item \textbf{The relatively low alpha (0.10) suggests strong phylogenetic signal in this trait-masting relationship.}
\item \textbf{Overall, the model provides no evidence that seed dormancy affects whether an angiosperm species is strong masting or not.}
\end{itemize}
\subsection{Reproductive type}
\textbf{Conifers}\\
\begin{itemize}
\item Dioecious species have a higher probability (70\%) of being strong masting species compared to monoecious species (44\%), but the difference is not significant.
\item \textbf{The very low alpha (0.007) suggests very strong phylogenetic signal in this trait-masting relationship.}
\item \textbf{Overall, the model provides no evidence that reproductive type affects whether a conifer species is strong masting or not.}
\end{itemize}
\textbf{Angiosperm}\\
\begin{itemize}
\item Dioecious species have 28\% probability of being strong masting species.
\item Monoecious species have higher probability (49\%), but this difference is not significant. Polygamous species have similar probability (26\%), but the difference is not significant.
\item \textbf{The relatively low alpha (0.11) suggests strong phylogenetic signal in this trait-masting relationship.}
\item \textbf{Overall, the model provides no evidence that reproductive type affects whether an angiosperm species is strong masting or not.}
\end{itemize}
\subsection{Drought tolerance}
\textbf{Conifers}\\
\begin{itemize}
\item High drought tolerance species have a high probability (91\%) of being strong masting species.
\item Moderate and low drought tolerance species have lower or higher probabilities (68\% and 96\% respectively), but differences are not significant.
\item \textbf{The moderate alpha (0.43) suggests moderate phylogenetic signal in this trait-masting relationship.}
\item \textbf{Overall, the model provides no evidence that drought tolerance affects whether a conifer species is strong masting or not.}
\end{itemize}
\textbf{Angiosperm}\\
\begin{itemize}
\item High drought tolerance species have a low probability (44\%) of being strong masting species.
\item Moderate and low drought tolerance species both have lower probabilities (40\% and 44\% respectively), but differences are not significant.
\item \textbf{The relatively low alpha (0.11) suggests strong phylogenetic signal in this trait-masting relationship.}
\item \textbf{Overall, the model provides no evidence that drought tolerance affects whether an angiosperm species is strong masting or not.}
\end{itemize}
\subsection{Seed weight}
\textbf{Conifers}\\
\begin{itemize}
\item \textbf{Seed weight has no significant effect on strong masting pattern for conifer species.}
\item \textbf{The moderate alpha (0.43) suggests moderate phylogenetic signal in this trait-masting relationship.}
\end{itemize}
\textbf{Angiosperm}\\
\begin{itemize}
\item \textbf{Higher seed weight significantly increases chance of being strong masting species for angiosperms, seed weight is a strong predictor of masting.}
\item \textbf{The relatively low alpha (0.063) suggests strong phylogenetic signal in this trait-masting relationship.}
\end{itemize}
\subsection{Fruit size}
\textbf{Conifers}\\
\begin{itemize}
\item \textbf{Fruit size has no significant effect on strong masting pattern for conifer species.}
\item \textbf{The moderate alpha (0.43) suggests moderate phylogenetic signal in this trait-masting relationship.}
\end{itemize}
\textbf{Angiosperm}\\
\begin{itemize}
\item \textbf{Fruit size has no significant effect on strong masting pattern for angiosperm species.}
\item \textbf{The relatively low alpha (0.17) suggests strong phylogenetic signal in this trait-masting relationship.}
\end{itemize}
\subsection{Seed size}
\textbf{Conifers}\\
\begin{itemize}
\item \textbf{Seed size has no significant effect on strong masting pattern for conifer species.}
\item \textbf{The moderate alpha (0.41) suggests moderate phylogenetic signal in this trait-masting relationship.}
\end{itemize}
\textbf{Angiosperm}\\
\begin{itemize}
\item \textbf{Seed size has no significant effect on strong masting pattern for angiosperm species.}
\item \textbf{The relatively low alpha (0.08) suggests strong phylogenetic signal in this trait-masting relationship.}
\end{itemize}
\subsection{Oil content}
\textbf{Conifers}\\
\begin{itemize}
\item Not enough data available for conifers.
\end{itemize}
\textbf{Angiosperm}\\
\begin{itemize}
\item \textbf{Oil content has no significant effect on strong masting pattern for angiosperm species.}
\item \textbf{The relatively low alpha (0.14) suggests relatively strong phylogenetic signal in this trait-masting relationship.}
\end{itemize}
\subsection{Leaf longevity}
\textbf{Conifers}\\
\begin{itemize}
\item \textbf{Leaf longevity has no significant effect on strong masting pattern for conifer species.}
\item \textbf{The moderate alpha (0.42) suggests moderate phylogenetic signal in this trait-masting relationship.}
\end{itemize}
\textbf{Angiosperm}\\
\begin{itemize}
\item \textbf{Leaf longevity has no significant effect on strong masting pattern for angiosperm species.}
\item \textbf{The relatively low alpha (0.006) suggests strong phylogenetic signal in this trait-masting relationship.}
\end{itemize}
\newpage
I also ran the common glm, using conifer and angiosperm as a fixed effect in the model, and I modified the results for better visualization:\\


\begin{figure}[H]
\centering
\includegraphics[width=\textwidth]{../output/figures/glmCat.pdf}
\caption{Predicted probability of being a strong masting species across categorical traits (pollination mode, seed dispersal, seed dormancy, mono/dio, and drought tolerance). Significance levels are indicated as follows: ns = not significant; * = p < 0.05.}
\end{figure}

\begin{itemize}
\item Pollination mode is a significant predictor of strong masting with wind-pollinated species have a significantly higher probability of being strong masting species and this pattern is consistent for both angiosperms and conifers.
\item Seed dispersal mode is a significant predictor of strong masting with species with biotic dispersed seeds have a significantly higher probability of being strong masting species and this pattern is consistent for both angiosperms and conifers
\item Reproductive type is a significant predictor of strong masting with monoecious species have a significantly higher probability of being strong masting species and this pattern is consistent for both angiosperms and conifers.
\item Seed dormancy and drought tolerance are not strong predictors of strong masting pattern for our studied species.
\end{itemize}

\newpage

\begin{figure}[H]
\centering
\includegraphics[width=\textwidth]{../output/figures/glmCon.pdf}
\caption{Effect sizes (log-odds) of continuous traits (fruit size, leaf longevity, oil content, seed size, and seed weight) on the probability of being a strong masting species. Points represent estimated coefficients, with error bars showing standard errors. Significance levels are indicated as follows: ns = not significant; * = p < 0.05.}
\end{figure}
\begin{itemize}
\item Fruit size is a significant predictor of strong masting for conifers, with log-odds of masting increaseing with increasing fruit size.
\item Oil content is a significant predictor of strong masting for conifers, with log-odds of masting increasing with higher oil content.
\item Seed size is a significant predictor of strong masting for conifers, with log-odds of masting increasing with bigger seeds.
\item Seed weight is a significant predictor of strong masting for both angiosperm and conifers, with log-odds of masting increasing with bigger seed weight.
\item Leaf longevity is not a strong predictor of strong masting pattern for either conifer or angiosperm.
\end{itemize}

\section*{Results for Continous Response Variable}
For the continuous response variable, I used the mean mast cycle obtained from Silvics and analyzed the data using PGLS. The results indicate that none of the traits is significantly related to mast cycle for either conifer or angiosperm species. There is no phylogenetic signal detected for any of the conifer traits, whereas weak phylogenetic signal is observed for seed dispersal, seed dormancy, reproductive type, drought tolerance, fruit size, and seed weight among angiosperm traits.
\begin{figure}[H]
\centering
\includegraphics[width=\textwidth]{../output/pglsConifer.pdf}
\end{figure}
\begin{figure}[H]
\centering
\includegraphics[width=\textwidth]{../output/pglsAngio.pdf}
\end{figure}
\end{document}
